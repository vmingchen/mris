hFS \ref{eurosys_hfs} also treats data differently based on their size and
type. Metadata and small files are stored separately in a log partition like
log-structured filesystem; data blocks of large regular files are stored in a
partition in a FFS-like fashion. Our work provides a different user interface,
which is an object store instead of a POSIX filesystem. Our work also focus
more on different block storage techniques, which is under the filesystem
layer.

\ref{fast_12_deindirection} also treats metadata and data differently, and
store them in virtual blocks and physical blocks respectively.

TableFS \ref{tablefs} also proposed to use NoSQL store metadata and small
files. However, TableFS is filesystem, which exposes POSIX interface to users,
whereas our store is an object store that not restricted by the POSIX
interface.  They do not privide extra functionalities like transaction and
journaling and they also did not consider using hybrid stroage media in their
work.

hFAD \ref{Seltzer09hfad} described an tag-based object store API that supports
full text search and it focused more on user interface. As our work is also
able to provide tagging and text searching of metadata, we are more focused on
workload specific performance optimization. Moreover, the design of hFAD is
above block level storage, whereas our work integrate hybrid block levle
stroage techniques into our storage system.

TODO: KVFS

TODO: GTSSL
GTSSL proposed an multi-tier archietecture of Tablet Sever Storage Layer.

TODO: All Sizes are Wrong

TODO: Rump file system

TODO: Gostor

Since multimedia files are primarily accessed sequentially, it may not be
necessary to provide for efficient random access to every large file.
\ref{evans2002study}.

\ref{kvworkload_sigmetrics} strong locality metrics, such as keys accessed many
millions of times a day, do not always suf- fice for a high hit rate; and there
is still room for efficiency and hit rate improvements in Memcached’s
implementation.  We found that the salient size characteristics follow
power-law distributions, sim- ilar to other storage and Web-serving systems

Why not cache in block level? Because block layer lacks the knowledge of
objects and files.  

It may even be worthwhile to investigate not caching large objects at all, to
increase overall hit rates \ref{kvworkload_sigmetrics}.

Forney et al. \ref{Forney2002fast} proposed storage aware caching for
heterogeneous storage systems. They made memory cache aware of the different
replacement costs and partitioned the cache for different storage devices.
However, their study is set in a different context which is a network-attached
disk system.  They did not consider data placement among different drives,
     which is an important strategy employed by our study.
