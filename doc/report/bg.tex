\section{Background}
\label{bg}

Background...

Typical length: 0 pages to 1.0.

Background and Related Work can be similar.
Most citations will be in this section.

1. Describe past work and criticize it, fairly.  Use citations
   to JUSTIFY your criticism!  Problem: hard to compare to YOUR
   work, b/c you've not yet described your work in enough
   detail.  Solution: move this text to Related Work at end of
   paper.

2. Describe in some detail, background material necessary to
   understand the rest of the paper.  Doesn't happen often,
   esp. if you've covered it in Intro.

Example, submit a paper to a storage conference: reviewers are
experts in storage.  Don't need to tell them about basic disk
operation.  But if your paper, say, is an improvement over an
already-advanced data structure (eg., COLA), then it'd make
sense to describe basic COLA algorithms in some detail.

Important: open the bg section with some "intro" text to tell
reader what to expect (so experienced readers can skip it).

If your bg material is too short, can fold it into opening of
'design' section.


\textbf{notes about picking a project}

Put every possible related citation you can! (esp. if conf.
doesn't count citations towards page size).

Literature survey:
- CiteSeer

- Google Scholar

- libraries

1. find a few relates paper

2. skim papers to find relevance

3. search for add'l related papers in Biblio.

4. reverse citation: use srch engines, to find
   newer papers that cite the paper you like.

5. "stop" when reach transitive closure

- then go off and read it.

- think about "how can I improve" and "what was so
  good about that paper".

- check future work for project ideas.

- go to talks \& conferences

Pick an idea:

- novelty vs. incremental (how big of an increment?)

- idea vs. practical implications
  (implemented? released? in use as OSS or commercial?)

- where to submit? good fit and match for quality.

- look at schedule of conferences: due dates \& result dates.

%%%%%%%%%%%%%%%%%%%%%%%%%%%%%%%%%%%%%%%%%%%%%%%%%%%%%%%%%%%%%%%%%%%%%%%%%%%%%%
%% For Emacs:
% Local variables:
% fill-column: 70
% End:
%%%%%%%%%%%%%%%%%%%%%%%%%%%%%%%%%%%%%%%%%%%%%%%%%%%%%%%%%%%%%%%%%%%%%%%%%%%%%%
%% For Vim:
% vim:textwidth=70
%%%%%%%%%%%%%%%%%%%%%%%%%%%%%%%%%%%%%%%%%%%%%%%%%%%%%%%%%%%%%%%%%%%%%%%%%%%%%%
% LocalWords:  SMR HDDs drive's SMRs
